%\documentclass{beamer}
\documentclass[xcolor=pdftex,dvipsnames,table]{beamer}
\usepackage{verbatim}


\usetheme{Luebeck}
\usecolortheme[named=RawSienna]{structure}
\setbeamerfont{frametitle}{family=\rmfamily,shape=\itshape}
%\setbeamertemplate{frametitle}[default][center]
\defbeamertemplate*{title page}{customized}[1][]
{
  \vspace{1.9cm}

% Content  
  \begin{minipage}{10cm}
  
  % Title and subtitle
  \begin{beamercolorbox}[center,#1]{title}
    \usebeamerfont{title}\inserttitle\par%
    \ifx\insertsubtitle\@empty%
    \else%
      \vskip0.25em%
      {\usebeamerfont*{subtitle}\usebeamercolor*[fg]{subtitle}\insertsubtitle\par}%
    \fi%
  \end{beamercolorbox}%
  
  \vskip1em\par
  
  % Display author(s)
  \begin{beamercolorbox}[center,#1]{author}
    \usebeamerfont{author}\insertauthor{}
  \end{beamercolorbox}
  
  \vskip1em\par

  % Institute
  \begin{beamercolorbox}[center,#1]{institute}
    \usebeamerfont{institute}\insertinstitute{}
  \end{beamercolorbox}
  
  \vskip1em%
  
  % Date
  \begin{beamercolorbox}[center,#1]{date}
    \usebeamerfont{date}\insertdate
  \end{beamercolorbox}\vskip0.5em

  \end{minipage}
  \hspace{0.6cm}
  \vfill

  \begin{flushleft}
  {\usebeamercolor[fg]{titlegraphic}\inserttitlegraphic\par}
  \end{flushleft}
  \hfill

%  \usebeamerfont{title}\inserttitle\par
%  \usebeamerfont{subtitle}\usebeamercolor[fg]{subtitle}\insertsubtitle\par
%  \bigskip
%  \usebeamerfont{author}\insertauthor\par
%  \usebeamerfont{institute}\insertinstitute\par
%  \usebeamerfont{date}\insertdate\par
%  \usebeamercolor[fg]{titlegraphic}\inserttitlegraphic
}

\title{Deferreds}
\subtitle{Strategy for Resource Reservation in the ITVM}
\author{David Buchmann}
\institute{Cisco Systems}

\date{\today}

\begin{document}

\maketitle

\begin{frame}[allowframebreaks]{Contents}
  \tableofcontents
\end{frame}

\section{ITVM}

\subsection{WTF?}
\begin{frame}
  \frametitle{ITVM: WTF?}
  \begin{columns}[cc]
  \column{0.5in}
  \includegraphics[scale=0.33]{itvm.jpg}
  \column{2.0in}
  \begin{itemize}
    \item Integrated Test Vending Machine
    \item $\leftarrow$ Not the institute
    \item Our in-house continous integration test running system
    \item Runs tests (regression, matchbox, etc)
  \end{itemize}
  \end{columns}
\end{frame}

\subsection{Test parameters}
\begin{frame}
  \frametitle{ITVM: What parameters affect reservation time?}
  \begin{itemize}
    \item Targets needed (the devices being tested) - hostname / type
    \item References needed (the devices used as references) -
      hostname / type
    \item Is matchbox - whether the test is a matchbox test
    \item Priority - default 0, the priority this test should run with
  \end{itemize}
\end{frame}

\section{Tests}
\subsection{Matchbox Tests (smoke tests)}
\begin{frame}
  \frametitle{Matchbox: Y u no run faster?}
  \begin{columns}[cc]
  \column{0.7 in}
  \includegraphics[scale=0.33]{matchbox.jpg}
  \column{2.3 in}
  \begin{itemize}
    \item The matchbox test (a.k.a our smoke test) takes a long time to start, why?
    \item Depending on product, the matchbox test
      needs different resources to run.
  \end{itemize}
  \end{columns}
\end{frame}

\section{Strategies}
\subsection{What do we want to prioritise?}
\begin{frame}
\frametitle{Prioritization}
\begin{columns}[cc]
\column{1.5in}
\includegraphics[scale=0.33]{priority.jpg}

\column{1.5in}
\begin{itemize}
  \item ITVMTest
  \item Matchbox
  \item Regression
\end{itemize}
\end{columns}
\end{frame}

\subsection{FIFO}

\begin{frame}
\frametitle{Naive Approach}
FIFO (first in - first out)
\begin{columns}[cc]
\column{2.0in}
\begin{itemize}
  \item Reserve (sorted by oldest)
  \item Stop when first test can't reserve all
  \item Pros: simple to implement
  \item Cons: 
    \begin{itemize}
      \item lots of resources left unused
      \item matchbox tests not prioritised
    \end{itemize}
\end{itemize}
\column{0.5in}
\includegraphics[scale=0.33]{fifo.png}
\end{columns}
\end{frame}

\begin{frame}
Example:
\begin{itemize}
  \item Pool has 3 snoopys and 3 falcons
  \item Task 1 requires 2 snoopies and 2 falcons
  \item Task 2 requires 1 snoopy and 2 falcons
  \item Task 3 requires 1 snoopy and 1 falcon and is a matchbox test
  \item Task 4 requires 3 snoopys
  \item Task 5 requires 1 snoopy and 1 falcon
  \item $\rightarrow$ Task 3 needs to wait for Task 1 and 2 to complete.
\end{itemize}
\end{frame}

\begin{frame}
\frametitle{Slightly less naive approach}
\begin{columns}[cc]
\column{2.0in}
Same strategy, queue sorted by
\begin{itemize}
  \item Priority (higher integer before lower)
  \item Is matchbox (true before false)
  \item Add time ascending (older before younger)
\end{itemize}
\column{0.5in}
\includegraphics[scale=0.50]{sort.png}
\end{columns}
\end{frame}

\begin{frame}
Example:
\begin{itemize}
  \item Pool has 3 snoopys and 3 falcons
  \item Task 3 requires 1 snoopy and 1 falcon and is a matchbox test
  \item Task 1 requires 2 snoopies and 2 falcons
  \item Task 2 requires 1 snoopy and 2 falcons
  \item Task 4 requires 3 snoopys
  \item Task 5 requires 1 snoopy and 1 falcon
  \item $\rightarrow$ Task 3 does not wait for anyone, but when Task 4
    is running we have a lock on all snoopys so the queue will grow
    very large - we would rather run this task when the queue is small
\end{itemize}
\end{frame}

\subsection{Current Approach}

\begin{frame}
\frametitle{The current approach: First-can-first-serve}
\begin{columns}[cc]
\column{0.3in}
\includegraphics[scale=0.20]{limited.jpg}
\column{2.2in}
\begin{itemize}

  \item Sorted queue from above

  \item Do \textbf{not} stop when first can not reserve
  \item Pros: Will always run the most possible tests in parallel
  \item Cons: Tests that require a lot of resources will find
    resources slower
\end{itemize}
\end{columns}
\end{frame}

\begin{frame}
Example:
\begin{itemize}
  \item Pool has 3 snoopys and 3 falcons
  \item Task 3 requires 1 snoopy and 1 falcon and is a matchbox test
  \item Task 1 requires 2 snoopies and 2 falcons
  \item Task 2 requires 1 snoopy and 2 falcons
  \item Task 4 requires 3 snoopys
  \item Task 5 requires 1 snoopy and 1 falcon
  \item $\rightarrow$ The chance that Task 4 runs when there is a long
    queue is smaller, so we can shorten the queue and prioritise
    getting done with as many tests as possible.
\end{itemize}
\end{frame}

\section{Statistics}
\begin{frame}
Friendly neighbourhood Statistics
\begin{itemize}
  \item Approx. 2500 tasks added/completed per day (266000 total)
  \item An avg. task requires 2.40 resources (with stddev. of 1.00)
  \item Avg. run time is 22 minutes
  \item Avg. time in queue is 74 minutes (stddev 4.5 hours)
    \begin{itemize}
      \item Matchbox: 46 minutes (stddev 2.5 hours)
      \item Others: 1 hour 46 minutes (stddev 5.8 hours))
    \end{itemize}
  \item 32.6 \% of all reservations are C20, 10.8 \% are E20 (94 C20s, 37 E20s in the pool)
\end{itemize}
\end{frame}
\end{document}
