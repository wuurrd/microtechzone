%\documentclass{beamer}
\documentclass[xcolor=pdftex,dvipsnames,table]{beamer}
\usepackage{verbatim}


\usetheme{Luebeck}
\usecolortheme[named=RawSienna]{structure}
\setbeamerfont{frametitle}{family=\rmfamily,shape=\itshape}
%\setbeamertemplate{frametitle}[default][center]
\defbeamertemplate*{title page}{customized}[1][]
{
  \vspace{1.9cm}

% Content  
  \begin{minipage}{10cm}
  
  % Title and subtitle
  \begin{beamercolorbox}[center,#1]{title}
    \usebeamerfont{title}\inserttitle\par%
    \ifx\insertsubtitle\@empty%
    \else%
      \vskip0.25em%
      {\usebeamerfont*{subtitle}\usebeamercolor*[fg]{subtitle}\insertsubtitle\par}%
    \fi%
  \end{beamercolorbox}%
  
  \vskip1em\par
  
  % Display author(s)
  \begin{beamercolorbox}[center,#1]{author}
    \usebeamerfont{author}\insertauthor{}
  \end{beamercolorbox}
  
  \vskip1em\par

  % Institute
  \begin{beamercolorbox}[center,#1]{institute}
    \usebeamerfont{institute}\insertinstitute{}
  \end{beamercolorbox}
  
  \vskip1em%
  
  % Date
  \begin{beamercolorbox}[center,#1]{date}
    \usebeamerfont{date}\insertdate
  \end{beamercolorbox}\vskip0.5em

  \end{minipage}
  \hspace{0.6cm}
  \vfill

  \begin{flushleft}
  {\usebeamercolor[fg]{titlegraphic}\inserttitlegraphic\par}
  \end{flushleft}
  \hfill

%  \usebeamerfont{title}\inserttitle\par
%  \usebeamerfont{subtitle}\usebeamercolor[fg]{subtitle}\insertsubtitle\par
%  \bigskip
%  \usebeamerfont{author}\insertauthor\par
%  \usebeamerfont{institute}\insertinstitute\par
%  \usebeamerfont{date}\insertdate\par
%  \usebeamercolor[fg]{titlegraphic}\inserttitlegraphic
}

\title{Deferreds}
\subtitle{Strategy for Resource Reservation in the ITVM}
\author{David Buchmann}
\institute{Cisco Systems}

\date{\today}

\begin{document}

\maketitle

\begin{frame}[allowframebreaks]{Contents}
  \tableofcontents
\end{frame}

\section{Introduction}
\subsection{Me... Zzzzz...}
\begin{frame}
  \begin{itemize}
    \item Work as a QA Tools developer
    \item Have worked in TANDBERG / CISCO for 3 years.
  \end{itemize}
\end{frame}

\section{ITVM Tasks}
\subsection{Matchbox Tests (smoke tests)}
\begin{frame}
  \frametitle{Matchbox: Y u no run faster?}
  \begin{columns}[cc]
  \column{0.5 in}
  \includegraphics[scale=0.30]{matchbox.jpg}
  \column{2.4 in}
  \begin{itemize}
    \item The matchbox test (a.k.a our smoke test) takes a long time to start, why?
    \item Depending on product, the matchbox test
      needs different resources to run.
  \end{itemize}
  \end{columns}
\end{frame}

\subsection{ITVM tests parameters}
\begin{frame}
  \frametitle{ITVM: What parameters affect planning?}
  \begin{itemize}
    \item Targets needed (the devices being tested) - hostname / type
    \item References needed (the devices used as references) -
      hostname / type
    \item Is matchbox - whether the test is a matchbox test
    \item Priority - default 0, the priority this test should run with
  \end{itemize}
\end{frame}

\subsection{ITVM: Test outcomes}
\begin{frame}
  \begin{columns}[cc]
  \column{1.5in}
  \includegraphics[scale=0.33]{brian.jpg}
  \column{1.5in}
  \begin{itemize}
    \item Passed
    \item Upload failure
    \item Setup failure
    \item Killed
  \end{itemize}
  \end{columns}
\end{frame}

\section{Strategies}
\subsection{What do we want to prioritise?}
\begin{frame}
\frametitle{Prioritization}
\begin{itemize}
  \item ITVMTest initialised tests
  \item Matchbox tests
  \item Regression tests
\end{itemize}
\end{frame}

\subsection{FIFO}
\begin{frame}
\frametitle{Naive Approach}
\begin{itemize}
  \item The first approach we attempted was FIFO (first in - first
    out).
  \item Get a sorted queue (by when added) of tests to run, stop when we do not have
    enough resources to run.
  \item Pros: simple to implement
  \item Cons: it's crap - lots of resources were left unused, and
    matchbox tests were not prioritised over regression that could
    complete during
\end{itemize}
\end{frame}

\begin{frame}
Example:
\begin{itemize}
  \item Pool has 3 snoopys and 3 falcons
  \item Task 1 requires 2 snoopies and 2 falcons
  \item Task 2 requires 1 snoopy and 2 falcons
  \item Task 3 requires 1 snoopy and 1 falcon and is a matchbox test
  \item Task 4 requires 3 snoopys
  \item Task 5 requires 1 snoopy and 1 falcon
  \item $\rightarrow$ Task 3 needs to wait for Task 1 and 2 to complete.
\end{itemize}
\end{frame}

\begin{frame}
\frametitle{Slightly less naive approach}
Same strategy, but sort the order of the queue to be run by
\begin{itemize}
  \item Priority
  \item Is matchbox
  \item When they were added 
\end{itemize}
\end{frame}

\begin{frame}
Example:
\begin{itemize}
  \item Pool has 3 snoopys and 3 falcons
  \item Task 3 requires 1 snoopy and 1 falcon and is a matchbox test
  \item Task 1 requires 2 snoopies and 2 falcons
  \item Task 2 requires 1 snoopy and 2 falcons
  \item Task 4 requires 3 snoopys
  \item Task 5 requires 1 snoopy and 1 falcon
  \item $\rightarrow$ Task 3 does not wait for anyone, but when Task 4
    is running we have a lock on all snoopys so the queue will grow
    very large - we would rather run this task when the queue is small
\end{itemize}
\end{frame}

\subsection{Current Approach}

\begin{frame}
\frametitle{The current approach: First-can-first-serve}
\begin{itemize}
  \item Go through the sorted queue, but do \textbf{not} stop when we
    do not have enough resources to run
  \item Pros: Will always attempt to run the most possible tests in parallel
  \item Cons: Tests that reserve a lot of resources will be less
    likely to run
\end{itemize}
\end{frame}

\begin{frame}
Example:
\begin{itemize}
  \item Pool has 3 snoopys and 3 falcons
  \item Task 3 requires 1 snoopy and 1 falcon and is a matchbox test
  \item Task 1 requires 2 snoopies and 2 falcons
  \item Task 2 requires 1 snoopy and 2 falcons
  \item Task 4 requires 3 snoopys
  \item Task 5 requires 1 snoopy and 1 falcon
  \item $\rightarrow$ The chance that Task 4 runs when there is a long
    queue is smaller, so we can shorten the queue and prioritise
    getting done with as many tasks as possible.
\end{itemize}
\end{frame}

\end{document}
